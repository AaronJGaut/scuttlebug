\documentclass[11pt, oneside]{article}   	% use "amsart" instead of "article" for AMSLaTeX format
\usepackage{geometry}                		% See geometry.pdf to learn the layout options. There are lots.
\geometry{letterpaper}                   		% ... or a4paper or a5paper or ... 
%\geometry{landscape}                		% Activate for for rotated page geometry
%\usepackage[parfill]{parskip}    		% Activate to begin paragraphs with an empty line rather than an indent
\usepackage{graphicx}				% Use pdf, png, jpg, or eps with pdflatex; use eps in DVI mode
								% TeX will automatically convert eps --> pdf in pdflatex		
\usepackage{amssymb}
\usepackage{amsmath}

\newcommand{\HRule}{\rule{\linewidth}{0.5mm}}

\title{A Two-Mass Oscillator}
\setlength\parindent{0pt}

\begin{document}

\begin{titlepage}
		\begin{center}
			\includegraphics[scale=0.2]{logo}\\[1cm]
			
			\textsc{\LARGE MAT 485 Project 1}\\[2cm]
			\textsc{\Large Cal Poly Pomona}\\[1cm]
			
		
			\HRule \\[0.4cm]
			{\huge \bfseries A Two-Mass Oscillator \\[0.4cm]}
			\HRule \\[2cm]
			
			\noindent
			\begin{minipage}{0.4\textwidth}
				\begin{flushleft}
					\large
					\emph{Authors:}\\
					Morgan Rupard \\ Aaron Gaut
				\end{flushleft}
			\end{minipage}
			\begin{minipage}{0.4\textwidth}
				\begin{flushright}
					\large
					\emph{Professor:}\\
					Dr. Jennifer Switkes
			\end{flushright}
			\end{minipage}
			
			\vfill
			
			{\large February $19^{\text{th}}$, 2016}
		\end{center}
	\end{titlepage}

\tableofcontents
\newpage

\section{Introduction}
In this paper we will model the two-mass oscillator system.
Consider the diagram shown in Figure 1.

\begin{figure}[h!]
\centering \includegraphics[scale=0.7]{sketch}
\caption{\label{sketch} The two-mass oscillator system}
\end{figure}

In this system there are two masses attached horizontally by a spring.
We will explore the scenario in which the spring's force is governed by Hooke's Law.
We will use a coordinate system relative to an arbitrary fixed origin.
The first object has mass $m_1$ and is positioned at $x_1$.
For the second mass, $m_2$ and $x_2$ are defined similarly.
The spring constant is determined by the parameter $k$, and the unstretched length of the spring is $L$.

The displacement by which the spring is stretch is given by $x_2 - x_1 -L$.
Therefore, by Hooke's Law, we can determine the acceleration of each mass,
\begin{align*}
m_1 \frac{d^2x_1}{dt^2} &= k(x_2 - x_1 - L), \\
m_2 \frac{d^2x_2}{dt^2} &= -k(x_2 - x_1 - L). \\
\end{align*}

From here, the following can be shown:
\begin{align}
&\frac{d^2}{dt^2}(m_1x_1 + m_2x_2) = 0, \\
&\frac{d^2z}{dt^2} = -k\left(\frac{1}{m_1}+\frac{1}{m_2}\right)z \text{, \;where } z = x_2-x_1-L,
\end{align}

In the following sections we will assume these initial conditions:
$$\begin{matrix}
x_1(0) = \alpha, & \dfrac{dx_1}{dt}(0) = \gamma \\ &\\
x_2(0) = \beta, & \dfrac{dx_2}{dt}(0) = \delta.
\end{matrix}$$
\section{Analytical Work}
We will now solve equation (1) analytically by integrating twice.
\begin{align*}
&\frac{d^2}{dt^2}(m_1x_1 + m_2x_2) = 0 \\
\implies & \frac{d}{dt}(m_1x_1 + m_2x_2) = C\\
\implies & m_1x_1 + m_2x_2 = Ct+D,
\end{align*}
for constants of integration $C$ and $D$. Now we impose the initial conditions.
We have the system of equations,
\begin{align*}
m_1\alpha + m_2\beta &= D, \\
m_1\gamma + m_2\delta &= C,
\end{align*}
and thus,
\begin{equation}
m_1x_1 + m_2x_2 = (m_1\gamma + m_2\delta)t + m_1\alpha + m_2\beta.
\end{equation}

Now, we will solve equation (2) using an eigenvalue approach.
If we do this then we will find that the roots of the characteristic polynomial are:

$$r = \pm \sqrt{k\left(\frac{1}{m_1}+\frac{1}{m_2}\right)}i$$

We will let $\displaystyle{\omega = \sqrt{k\left(\frac{1}{m_1}+\frac{1}{m_2}\right)}}$. This implies that the the general solution to equation (2) is:

$$z(t) = c_1\cos{(\omega t)}+c_2\sin{(\omega t)}$$

We will now impose the initial conditions in order to find $c_1$ and $c_2$.
We will also back-substitute $x_1$ and $x_2$ into the equation, which results in:

\begin{equation}
x_2-x_1-L = \left(\beta - \alpha - L\right)\cos{\left(\omega t\right)}+\frac{\delta - \gamma}{\omega}\sin{(\omega t)}
\end{equation}

\section{Numerical Work}
We used the MATLAB function \texttt{fsolve}, a nonlinear implicit equation solver, together with the analytically derived equations (3) and (4), to explore the system.
We used two sets of initial conditions.

\subsection{Sanity Checks}
First we will verify our numerical work by 

\subsection{Volumeless Entity: The Spring Singularity}
In this section we will explore what happens when we choose a set of initial conditions that maximizes the displacement of each individual mass.
When this happens the two curves can be seen touching at a single point along the center of mass line.
This is not physically possible because it implies that the spring has popped out of existence or has phased into the two masses.
Even though it is not possible it is still interesting to look at and makes a really pretty picture.
It turns out that it is a little bit difficult to find the correct constants for this to happen because we need the total maximum amplitude of each curve to be $L$, which would mean solving this equation for each constant:

$$\sqrt{(\beta - \alpha -L)^2 + \left(\frac{\delta - \gamma}{\omega}\right)^2}=L$$

This is incredibly difficult, but we can impose some requirements that makes this a simpler task such as making one of the terms under the square root $0$:

$$\delta - \gamma = 0 \implies \beta - \alpha = 2L$$

If we do this using the parameters

\begin{center}

\begin{tabular}{| c | c | c | c | c | c | c | c |}

\hline

$L$ & $\alpha$ & $\beta$ & $\gamma$ & $\delta$ & $m_1$ & $m_2$ & $k$ \\

\hline

 9 & 0 & 18 & 2 & 2 & 10 & 5 & 4\\

\hline

\end{tabular}

\end{center}

we then get a plot that looks like: \\

\includegraphics[scale=0.3]{spring_sing}


\section{Discussion}

\end{document}  